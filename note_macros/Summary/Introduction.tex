\section{Introduction}\label{sec:intro}
%\fixme{Add Introduction}
The CMS experiment is expected to make major new discoveries at the LHC
 and make precision measurements of the properties of the fundamental
 particles and interactions. The key to these discoveries and
 measurements is the ability to trigger on, and reconstruct, muons with
 high efficiency. The muon reconstruction algorithms have been designed
 to achieve these goals and this note decscribes the current performance
 of the algorithms. The analysis presented here is \it ideal \rm in that
 it does not include effects such as miscalibration or
 detector inefficiencies, except those caused by the detector
 geometry. Similarly actual event environments or beam induced
 backgrounds are not studied.  The performance of these
 algorithms has been evaluated using the full detector simulation with a
 magnetic field of 4 Tesla. The performance has been tested using
 samples of single muons generated with different values of \pt{} and
 flat distributions in $\eta$ and $\phi$ and in the presence of more
 than one muon and with non-flat distributions.  The categories of
 reconstruction analyzed are \bi
\item Stand-alone reconstruction: this just uses hits in the muon detectors
\item Global Reconstruction: this starts with the muon segment information and then adds tracker information
\item Tracker Muon reconstruction: this starts with tracks found in
  the inner tracker and identifies them as muon 
by matching expected information from the calorimeters and muon system.
\ei 
In all cases the \it beam \rm spot position is used as a constraint.
 
\section{Muon Reconstruction Efficiencies}\label{sec:efficiencies}
In \qfig{fig:sta_sim_eff_inclusive} the total efficiency for the
stand-alone reconstruction ($\varepsilon_{sa}$) is shown. The loss in
efficiency at $|\eta| \sim 0.3$ is due to a geometrical effect, since in
that region there is a discontinuity between the central wheel and its
neighbours.  The dips in the $0.8 < |\eta| < 1.2$ region are due to
failures in the seed-finding algorithm which may be recovered although
that region is known to be problematic as in that region DT and CSC
segments are used together to estimate the seed state.

The inner tracker is less affected by multiple scattering and energy
loss than the muon system. Moreover the magnetic field in the tracker
volume is homogeneous and almost constant. In
\qfigand{fig:tk_sim_eff_inclusive}{fig:tkmu_sim_eff_inclusive} the
efficiency as a function of pseudorapidity and $\phi$, for different
\pt{} samples, is shown. The integrated efficiency is almost constant
for all \pt{} values (\qfig{fig:tk_sim_eff_inclusive}) and its value is
above 99.5\%.  The dip at $\eta\simeq 0$ is due to the tracker
geometry. The tracker is made of two half-barrels joined together, and
the junction surface is at $\eta = 0$.

Global reconstruction requires the matching of reconstructed tracks in
both the muon system and the tracker and the overall efficiency is a
product of the stand-alone, the tracker track and the matching
efficiencies.The results are shown in the following figures:
(\qfig{fig:glb_sim_eff_inclusive}) shows the global efficiency
$\varepsilon_{glb}$, $\varepsilon_{glb,sa} =
\varepsilon_{glb}/\varepsilon_{sta}$, (\qfig{fig:glb_sta_eff_inclusive})
shows $\varepsilon_{glb,tk}$ , and (\qfig{fig:glb_tk_eff_inclusive})
shows $\varepsilon_{glb}/\varepsilon_{tk}$.

The efficiency plot shown in \qfig{fig:glb_sim_eff_inclusive} exhibit
a number of lower efficiency structures which correspond to
discontinuities in the geometry of the detector. These
are as follows: \bi
   \item $\eta \simeq 0$: junction surface between the two
     tracker barrels;
   \item $|\eta| \simeq 0.3$: inter-space between the DT central wheel
     and its neighbours;
   \item $0.8<|\eta|<1.2$: problematic region for seed
     estimation (DT and CSC overlap);
   \item $|\eta| \simeq 1.8$: problematic region for tracker track
     reconstruction (transition from TID to TID/TEC subsystem); 
   \item $\phi \simeq 1.2$: barrel inactive region (chimney),
                            because of instrumentation services;
   \item periodic structure in $phi$: loss in efficiency in the
     stand-alone muon reconstruction due to muons which escape in the
     space between two adjacent sectors or chambers in CSCs (although all
     chambers overlap in $\phi$, except those in ME1/3).
     \ei
%

The $\varepsilon_{matching}$ is shown in
(\qfig{fig:glb_matching_eff_inclusive}) as a function of $\eta$ and
$\phi$ and the integrated efficiencies are shown in \qtab{tab:glb_eff}.
The matching efficiency is above 99\% over all the \pt{} spectra down to
5~\GeVc, where it is around 91\%. This drop in efficiency for low \pt{}
is due to muon seed parameters which are poorly estimated and this
directly affects the stand-alone muon reconstruction and the subsequent
matching with the tracker tracks.  This efficiency can be improved by
tuning the seed parameters for low \pt{} muons.  $\varepsilon_{glb,sta}$
shows that the stand-alone muon efficiency dominates the final muon
reconstruction efficiency.


The \qfig{fig:all_eff_unbinned} shows a direct comparison of the
 reconstruction efficiencies: $\varepsilon_{seed}$, $\varepsilon_{sa}$,
 $\varepsilon_{tk}$ and $\varepsilon_{glb}$, as a function of \pt{},
 $\eta$ and $\phi$.  The values are almost the same for
 $\pt{} \geq 10~\GeVc{}$. The structure visible in the efficiency as a
 function of $\phi$, for the seed, stand-alone and global
 reconstruction, is due to the $\phi$-acceptance of the muon system.

The efficiency as a function of the muon transverse momentum is shown in
(\qfig{fig:all_eff_unbinned}).  The muon reconstruction efficiency increases
up to a plateau which is approximately constant from $8~\GeVc{}$ to
$1~\TeVc{}$ (starting from $\pt=5~\GeVc{}$ more than 50\% of muons are
reconstructed).  At TeV momenta the muon reconstruction efficiency
decreases slowly, due to the effects of bremsstrahlung on finding
correct seeds.

The tracker efficiency is stable above $1~\GeVc{}$ so the loss in
efficiency at low \pt{} is due to the muon identification efficiency.
The conclusion from the analysis is that high reconstruction efficiency
in the tracker together with the muon spectrometer ensure a robust muon
identification system with minimal losses.

\section{Charge Identification}\label{sec:chargeId}
The charge identification probability has been evaluated for the
different stages of the reconstruction. In 
%\qfig{fig:seed_charge_inclusive}, 
\qfig{fig:sta_charge_inclusive}, \qfig{fig:tk_charge_inclusive}, and~\qfig{fig:glb_charge_inclusive} the probability to assign the correct charge
is shown as a function of $\eta$ and $\phi$.


\section{\pt Resolution from Silicon Tracker, Muon System, and Combined System}
\label{sec:fullSystem}
Since the pattern recognition of selecting appropriate hits has already
been performed during the reconstruction of the tracker track and stand-alone
muon track, there is no additional pattern recognition to be done for
the global muon track fit.  The default global muon algorithm simply
combines the collection of tracker hits corresponding to the chosen
tracker track with the collection of muon hits corresponding to the
stand-alone muon track. However it is also possible to combine only a
subset of the hits for the global fit. Choosing a subset of the muon
hits provides a better reconstruction resolution for high energy muons,
when the measurements in the muon system are frequently contaminated by
electromagnetic showers.

The contribution due to multiple scattering is the \emph{flat} component
on the resolution curves vs momentum, this appears below $100~\GeVc$ for
the Muon system and below $50~\GeVc$ for the Tracker Simple calculation
for the magnitude shows that for the \emph{muon system} (excluding the
tracker region) with $L/X_o=220$, and average of ${\bf{B}}=2$ T and
$L=(7-1)$m we expect a momentum resolution of 5\%, which is within 20\%
of what is seen; recalling that the formula used is accurate within 11\%
for $L/X_o\leq 100$. For the \emph{tracker system} $L/X_o \cong 0.5$ and
uniform ${\bf{B}}=4$ T and $L=1$m we expect a momentum resolution of
0.9\%, this is roughly what is seen.

At higher momenta the \emph{intrinsic} resolution due to the measurements, 
being proportional to the momentum, becomes significant. This is a 
linear increase in the tracker above $50~\GeVc$, and later becomes of the same order of magnitude in the Muon system at 2 TeV in the barrel. This effect occurs
in the muon system later in momenta because of the larger amount of
material of an additional 60 radiation lengths in the end-caps and
somewhat less in the overlap.

The effect of adding one muon hit with very precise resolution has
little or not effect on the combined fit. The reason has to do with the
fact that we update the stand-alone trajectory to originate from the
vertex, so the stand-alone fit and the stand-alone + 1 tracker hit are
essentially identical; The resolution as a function of the number of
tracker hits added to the muon system is more significant below $\pt
\leq 10 \GeVc$ and above $\pt \geq 100 \GeVc$, where the improvements
arise due to the multiple scattering component for the low momentum case
and to the intrinsic resolution of the muon ( + tracker) system in the
$\pt \geq 100 \GeVc$ range.

An inherent feature for all plots is the
sudden decrease in resolution above $|\eta|\geq 1.5$. The two main
contributions for this effect are the increase in material in the
end-caps and the steady decrease of the magnetic field. The
\emph{stand-alone} fit directly reflects the distribution of the
material at low momentum (neglecting non-linear magnetic field effects),
we can infer this another way, given that the fact that the resolution
curves cross for momenta below a few hundred GeVs, shows that the
resolution component linear in momentum is small compared to the
resolution component due to multiple scattering.

Adding hits from the \emph{tracker} to the stand-alone fit has the effect
of \emph{leveling out} the features of the curves due to material
effects.  As more hits are added, the weight of the tracker hits
can eventually separate the stand-alone curves at low momentum.

At $|\eta| = 1.5$ there is a minimum in the material, this directly
translates into a minimum for resolution curves.


\section{Effect of Misalignment}\label{sec:misalignment}
To properly reconstruct muons, the positions and orientations of all
elements in the silicon tracker and the muon system need to be
well-known relative to one another.  Misalignment, by which we mean an
incorrect assumption about the geometry of the detectors, causes
errors in the direction and curvature of reconstructed tracks, both of
which have a degrading effect on reconstructed masses.  To minimize
these errors, data from dedicated hardware alignment systems and from
the tracks themselves are used to identify the true positions of the
detector elements in a common coordinate system, but there will always
be some residual error.  In this section, we calculate the error in
muon \pt and dimuon masses which would be incurred under a variety
of realistic alignment scenarios.  The scenarios have been derived
from experiences with the hardware alignment systems at the TIF
(Tracker Integration Facility, for the silicon tracker), global runs
in the spring of 2008 (for the muon alignment system), and the CSA08
computing exercise (for track-based alignment of both systems).

\qfig{misalignment:curvature_resolution} presents the curvature
($1/\pt$) resolution of muon tracks over three orders of magnitude in
\pt.  Each series of points represents a different alignment
scenario: a perfectly-aligned detector (what one would get from the
constants database using an ``IDEAL\_V$n$'' tag), the result of the
$10~\pbinv$ CSA08 exercise (a ``CSA08\_S156'' tag), a
randomly-generated estimate of alignment quality after $10~\pbinv$
(``10PB\_V$n$''), used in the generation of some Monte Carlo samples,
and a randomly-generated estimate of alignment quality with only
hardware data (``STARTUP\_V$n$'').  Muons were selected from physics
samples: $\JPsi$ (below 5~\GeVc) using the calorimeter only (caloMuons),
$Z$ (5--50~\GeVc) using full global muons (globalMuons), and Sequential
Standard Model $Z'$ (above 50~\GeVc) using the first hit on every
station (TeVmuon/firstHit) to avoid losses in resolution due to
showering.

To interpret this plot, note first that the fractional uncertainty in
curvature ($1/\pt$) for a given track is equal to the fractional
uncertainty in transverse momentum (\pt) for that track, so we can
read the
\[(\mbox{reconstructed} - \mbox{generated}) / \mbox{generated} \]
curvature resolution as transverse momentum resolution.


%%% Local Variables: 
%%% mode: latex
%%% TeX-master: "MuonReco"
%%% End: 
